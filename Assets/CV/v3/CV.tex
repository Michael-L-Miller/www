% Document history
%
% Curriculum vitae for Michael Lawrence Miller
% $Id: CV-MLMiller.tex, v3.6 07-March-2018
% Based on "Curriculum vitae for Joseph D. Monaco"
% jdmonaco-cv.tex, downloaded from https://github.com/jdmonaco/cv
%
% For the initial template h/t Matthew M. Boedicker <mboedick@mboedick.org> http://mboedick.org
% ===============================================================================================

\documentclass[10pt]{article}
\usepackage{fullpage}
\usepackage{palatino}
\usepackage[hidelinks]{hyperref}
\usepackage{multirow}
\usepackage{tabu}
\usepackage{fancyhdr}
\usepackage[utf8]{inputenc}

\pagestyle{fancy}
\textheight=9.0in
\raggedbottom
\raggedright
\setlength{\tabcolsep}{0in}
\addtolength{\voffset}{-.6in}
\addtolength{\textheight}{.9in}
\addtolength{\hoffset}{-.5in}
\addtolength{\textwidth}{.95in}

\fancyhf{}
\renewcommand{\footrulewidth}{0.5pt}
\renewcommand{\headrulewidth}{0pt}
\addtolength{\headwidth}{1in}
\fancyfoot[C]{\thepage}
\fancyfoot[L]{\textit{Miller, ML}}
\fancyfoot[R]{Updated 07-March-2018, v3.6}

\begin{document}

\begin{tabular*}{7.5in}{c@{\extracolsep{\fill}}rl}
\hline\\[0.02in]
\textsc{\textbf{\Large Michael Lawrence Miller, Ph.D.}}    & \textsc{Phone / Fax}      & \texttt{516-244-0672 / 707-986-4043} \\
\multirow{2}{*}{\large }                                & \textsc{E-mail}      & \href{mailto:michael.miller@icahn.mssm.edu}{\texttt{michael.miller@icahn.mssm.edu}} \\
{\small Icahn School of Medicine at Mount Sinai}    & \textsc{Web}        & \href{http://millem14.u.hpc.mssm.edu}{\texttt{millem14.u.hpc.mssm.edu}} \\
{\small One Gustave L. Levy Place}                & \textsc{Google Scholar}      & \href{https://scholar.google.com/citations?user=7EVp2IkAAAAJ&sortby=pubdate}{\texttt{7EVp2IkAAAAJ}} \\  
{\small New York, NY, 10029, USA}                      & \textsc{Doximity}   & \href{https://www.doximity.com/profile/6396828}{\texttt{6396828}} \\[0.1in]
\hline
\end{tabular*}

\vspace{0.25in}

{\large \textbf{Education}}
\begin{itemize}
  \item 
  \begin{tabular*}{7.1in}{l@{\extracolsep{\fill}}r}
    \textbf{Icahn School of Medicine at Mount Sinai} & New York, NY \\
    Ph.D. in Neuroscience (2016), M.D. (expected May 2018) & 2010--present \\
    Dissertation Title: \textit{Genetic and epigenetic risk factors of opiate abuse} \\
  \end{tabular*}
  
  \item 
  \begin{tabular*}{7.1in}{l@{\extracolsep{\fill}}r}
    \textbf{Binghamton University} & Binghamton, NY \\
    B.S. in Integrative Neuroscience, B.S. in Biochemistry, & 2006--10 \\
    Dean's Certificate in evolutionary studies  \\
    \textit{Summa cum laude}\\
  \end{tabular*}
\end{itemize}

\vspace{0.1in}
{\large \textbf{Research Experience}}
\begin{itemize}

\item
  \begin{tabular*}{7.1in}{l@{\extracolsep{\fill}}r}
    \textbf{Icahn School of Medicine at Mount Sinai} & New York, NY \\
    Fishberg Department of Neuroscience, Friedman Brain Institute & 2012--present \\
    Doctoral Candidate, Medical Scientist Training Program \\
    Dissertation Advisor: Yasmin L. Hurd, Ph.D. \\
  \end{tabular*}
\item
  \begin{tabular*}{7.1in}{l@{\extracolsep{\fill}}r}
    \textbf{Brookhaven National Laboratories} & New York, NY \\
    Department of Medicine & 2008--10 \\
    Summer Undergraduate Researcher \\
    Supervisor: Panayotis (Peter) K. Thanos, Ph.D. \\
  \end{tabular*}
\item
    \begin{tabular*}{7.1in}{l@{\extracolsep{\fill}}r}
    \textbf{Binghamton University} & Binghamton, NY \\
    Department of Biological Sciences & 2007--10 \\
    Undergraduate Researcher \\
    Honor's Thesis Advisor: Anne B. Clark, Ph.D. \\
  \end{tabular*}
\end{itemize}

\vspace{0.1in}
{\large \textbf{Grants and Fellowships}}
\begin{itemize}

\item
  \begin{tabular*}{7.1in}{l@{\extracolsep{\fill}}r}
    \textbf{Ruth L. Kirschstein National Research Service Award Fellow} & 2015--present \\
    National Institute on Drug Abuse, National Institutes of Health & \$171,370 \\
    Project title: \textit{Identifying distinct striatonigral and striatopallidal disturbances} (\href{https://projectreporter.nih.gov/project_info_description.cfm?aid=8836235}{F30-DA-038954}) \\
  \end{tabular*}
\item
  \begin{tabular*}{7.1in}{l@{\extracolsep{\fill}}r}
    \textbf{Barry M. Goldwater Scholar in Mathematics, Science and Engineering} & 2009--10 \\
    The Barry M. Goldwater Foundation & \$7,500 \\
  \end{tabular*}
\end{itemize}

\vspace{0.1in}
{\large \textbf{Publications}}

\begin{description}

\item Chadwick B, \textbf{Miller ML}, Dickstein DL, Purushothaman I, Egervari G, Rahman T, Tessereau C, Hof PR, Roussos P, Shen L, Baxter MG, \& Hurd YL. (in revision). Adolescent exposure to $\Delta$\textsuperscript{9}-tetrahydrocannabinol alters the transcriptional trajectory and dendritic architecture of prefrontal pyramidal neurons.
\item Michaelides M, \textbf{Miller ML}, DiNieri JA, Gomez JL, Schwartz E, Egervari G, Wang G-J, Mobbs CV, Volkow ND, \& Hurd YL. (2017). Dopamine D2 receptor signaling in the nucleus accumbens comprises a metabolic-cognitive brain interface regulating metabolic components of glucose reinforcement. \textit{Neuropsychopharmacology}, \textit{42}, 2365--2376. \href{https://doi.org/10.1038/npp.2017.112}{doi:~10.1038/npp.2017.112}.
\item \textbf{Miller ML}, Ren Y, Szutorisz H, Warren NA, Tessereau C, Egervari G, Mlodnicka A, Kapoor M, Chaarani B, Morris CV, Schumann G, Garavan H, Goate AM, Bannon MJ, IMAGEN Consortium, Halperin JM, \& Hurd YL. (in press). Ventral striatal regulation of \textit{CREM} mediates impulsive action and drug addiction vulnerability. \textit{Molecular Psychiatry}. \href{https://doi.org/10.1038/mp.2017.80}{doi:~10.1038/mp.2017.80}.
\item \textbf{Miller ML} \& Hurd YL. (2017). Testing the Gateway Hypothesis. \textit{Neuropsychopharmacology}, \textit{42}, 985--986. \href{https://doi.org/10.1038/npp.2016.279}{doi:~10.1038/npp.2016.279}.
\item Egervari G, Jutras-Aswad D, Landry J, \textbf{Miller ML}, Anderson SA, Michaelides M, Jacobs MM, Peter C, Yiannoulos G, Liu X, \& Hurd YL. (2016). A functional 3’UTR polymorphism (rs2235749) of Prodynorphin alters microRNA-365 binding in ventral striatonigral neurons to influence novelty seeking and positive reward traits. \textit{Neuropsychopharmacology}, \textit{10}, 2512--2520. \href{https://doi.org/10.1038/npp.2016.53}{doi:~10.1038/npp.2016.53}.
\item Thanos PK, Michaelides M, Subrize M, \textbf{Miller ML}, Bellezza R, Cooney RN, Leggio L, Wang G-J, Rogers AM, Volkow ND, \& Hajnal A. (2015). Roux-en-Y gastric bypass alters brain activity in regions that underlie reward and taste perception. \textit{PLoS One}, \textit{10}, e0125570. \href{https://doi.org/10.1371/journal.pone.0125570}{doi:~10.1371/journal.pone.0125570}.
\item \textbf{Miller ML}, Chadwick B, Morris CV, Michaelides M, \& Hurd YL. (2015). Cannabinoid-Opioid Interactions. In P. Campolongo \& L. Fattore (Eds.) Cannabinoid Modulation of Emotion Memory and Motivation (pp 393--407) New York NY: Springer New York. \href{https://doi.org/10.1007/978-1-4939-2294-9_15}{doi:~10.1007/978-1-4939-2294-9\_15}.
\item Maze I, Chaudhury D, Dietz DM, von Schimmelmann M, Kennedy PJ, Lobo MK, Sillivan SE, \textbf{Miller ML}, Bagot RC, Sun H, Turecki G, Neve RL, Hurd YL, Shen L, Han M-H, Schaefer A, \& Nestler EJ. (2014). G9a influences neuronal subtype specification in striatum. \textit{Nature Neuroscience}, \textit{17}, 533--539. \href{https://doi.org/10.1038/nn.3670}{doi:~10.1038/nn.3670}.
\item Szutorisz H, DiNieri JA, Sweet E, Egervri G, Michaelides M, Carter JM, Ren Y, \textbf{Miller ML}, Blitzer RD, \& Hurd YL. (2014). Parental THC exposure leads to compulsive heroin-seeking and altered striatal synaptic plasticity in the subsequent generation. \textit{Neuropsychopharmacology}, \textit{39}, 1315--1323. \href{https://doi.org/10.1038/npp.2013.352}{doi:~10.1038/npp.2013.352}.
\item Hurd YL, Michaelides M, \textbf{Miller ML}, \& Jutras-Aswad D. (2014). Trajectory of adolescent cannabis use on addiction vulnerability. \textit{Neuropharmacology}, \textit{76}, 416--424. \href{https://doi.org/10.1016/j.neuropharm.2013.07.028}{doi:~10.1016/j.neuropharm.2013.07.028}.
\item Chadwick B, \textbf{Miller ML}, \& Hurd YL. (2013). Cannabis use during adolescent development: susceptibility to psychiatric illness. \textit{Frontiers in Psychiatry}, \textit{4}, 129. \href{https://doi.org/10.3389/fpsyt.2013.00129}{doi:~10.3389/fpsyt.2013.00129}.
\item Michaelides M, \textbf{Miller ML}, Subrize M, Kim R, Robison L, Hurd YL, Wang G-J Volkow ND, \& Thanos PK. (2013). Limbic activation to novel versus familiar food cues predicts food preference and alcohol intake. \textit{Brain Research}, \textit{1512}, 37--44. \href{https://doi.org/10.1016/j.brainres.2013.03.006}{doi:~10.1016/j.brainres.2013.03.006}.
\item \textbf{Miller ML}, Gallup AC, Vogel AR, Vicario SM, \& Clark AB. (2012). Evidence for contagious behaviors in budgerigars (\textit{Melopsittacus undulatus}): An observational study of yawning and stretching. \textit{Behavioural Processes}, \textit{89}, 264--270. \href{https://doi.org/10.1016/j.beproc.2011.12.012}{doi:~10.1016/j.beproc.2011.12.012}.
\item \textbf{Miller ML}, Gallup AC, Vogel AR, \& Clark AB. (2012). Auditory disturbances promote temporal clustering of yawning and stretching in small groups of budgerigars (\textit{Melopsittacus undulatus}). \textit{Journal of Comparative Psychology}, \textit{126}, 324--328. \href{https://doi.org/10.1037/a0026520}{doi:~10.1037/a0026520}.
\item \textbf{Miller ML}, Gallup AC, Vogel AR, \& Clark AB. (2010). Handling-stress initially inhibits but then potentiates yawning in budgerigars (Melopsittacus undulatus). \textit{Animal Behaviour}, \textit{80}, 615--619. \href{https://doi.org/10.1016/j.anbehav.2010.05.018}{doi:~10.1016/j.anbehav.2010.05.018}.
\item Gallup AC, \textbf{Miller ML}, \& Clark AB. (2010). The direction and range of ambient temperature change influences yawning in budgerigars (\textit{Melopsittacus undulatus}). \textit{Journal of comparative Psychology}, \textit{124}, 133--138. \href{https://doi.org/10.1037/a0018006}{doi:~10.1037/a0018006}.
\item Gallup AC, \textbf{Miller ML}, \& Clark AB. (2009). Yawning and thermoregulation in budgerigars: Science as an incremental process. \textit{Animal Behaviour}, \textit{78}, e3--e5. \href{https://doi.org/10.1016/j.anbehav.2009.09.017}{doi:~10.1016/j.anbehav.2009.09.017}.
\item Gallup AC, \textbf{Miller ML}, \& Clark AB. (2009). Yawning and thermoregulation in parakeets (\textit{Melopsittacus undulatus}). \textit{Animal Behaviour}, \textit{77}, 109--113. \href{https://doi.org/10.1016/j.anbehav.2008.09.014}{doi:~10.1016/j.anbehav.2008.09.014}.
\end{description}

\vspace{0.1in}

{\large \textbf{Presentations}}
\begin{description}
\item \textbf{Talks}
\item[\quad] \textbf{Miller ML}. Transcriptional microdissection of striatal neurocircuits: applications for studying substance use disorder. Undergraduate Research Symposium in Biological, Chemical, Structural, and Computational Sciences, Icahn School of Medicine at Mount Sinai, New York, NY, 17 September 2016.
\end{description}

\begin{description}
\item \textbf{Posters}

\item[\quad] \textbf{Miller ML}, Ye F, Tome-Garcia J, Zhang DY, \& Tsankova NM. Assessment of focal gene amplifications and EGFRvIII mutation in glioblastoma through next-generation sequencing. 93\textsuperscript{rd} Annual Meeting of the American Association of Neuropathologists, Hyatt Regency Orange County, Garden Grove, CA, 8--11 June 2017.

\item[\quad] \textbf{Miller ML}, Ren Y, Szutorisz H, Egervári G, Morris CV, Sperry J, \& Hurd YL. Striatal \textit{CREM} mediates impulsive action and associates with fatal overdose in heroin abusers. 7\textsuperscript{th} Annual Neuroscience Retreat at the Icahn School of Medicine at Mount Sinai, The New York Academy of Medicine, New York, NY, 13 May 2015.

\item[\quad] \textbf{Miller ML}, Chadwick B, \& Hurd YL. Identifying distinct striatonigral and striatopallidal disturbances after developmental cannabis exposure: pathway-specific changes in miRNAs and their contribution to heroin addiction. 6\textsuperscript{th} Biennial Gordon Research Conference on Cannabinoid Function in the CNS, Renaissance Tuscany Il Ciocco, Barga LU, Italy, 24--29 May 2015.

\item[\quad] \textbf{Miller ML} \& Hurd YL. Identifying distinct striatonigral and striatopallidal disturbances after developmental cannabis exposure: pathway-specific changes in miRNAs and their contribution to heroin addiction. 14\textsuperscript{th} Annual Medical-Scientist Training Program Retreat at the Icahn School of Medicine at Mount Sinai, Honors Haven Resort, Ellenville, NY, 5--7 September 2014.

\item[\quad] \textbf{Miller ML} \& Hurd YL. The impact of adolescent THC on mesolimbic microRNA regulation. 5\textsuperscript{th} Biennial Gordon Research Conference on Cannabinoid Function in the CNS, Waterville Valley, NH, 4--9 August 2013.
\end{description}

\vspace{0.1in}
{\large \textbf{Mentorship and Teaching}}

\begin{itemize}

\item
  \begin{tabular*}{6.5in}{l}
    \textbf{Teaching Assistant for Brain \& Behavior} \\
    Fall 2012 through Fall 2015, Directed by Carrie Ernst, M.D. \\
  \end{tabular*}
\item
  \begin{tabular*}{6.5in}{l}
    \textbf{Teaching Assistant for Principles of Neurobiology: Systems Neuroscience} \\
    Winter 2013, Directed by Elizabeth Cropper, Ph.D. \\
  \end{tabular*}
\item
  \begin{tabular*}{6.5in}{l}
    \textbf{Teaching Assistant for General Pathology} \\
    Winter 2012, Directed by Margret S. Magid, M.D. \\
  \end{tabular*}
\item
  \begin{tabular*}{6.5in}{l}
    \textbf{Undergraduate Teaching Assistant for Cellular Neurobiology} \\
    Fall 2008 and Fall 2009, Directed by Carol I. Miles, Ph.D. \\
  \end{tabular*}  
\item
  \begin{tabular*}{6.5in}{l}
    \textbf{Mentored summer undergraduate interns and Master's students} \\
    J Rodriguez (2013), N Downer (2014), Y Valentine (2015) and T Rahman (2015--2016).\\
    Directed by Yasmin L. Hurd, Ph.D. \\
  \end{tabular*}  
\end{itemize}

\vspace{0.1in}
{\large \textbf{Awards}}

\begin{itemize}
  \item Icahn School of Medicine Medical Student Travel Award (Spring 2017), \$500
  \item Icahn School of Medicine Graduate Student Travel Award (Spring 2015, Spring 2017), \$400 each
  \item Society for Neuroscience Graduate Student Travel Award (2013)
  \item Gordon Research Conference (GRC) Graduate Student Poster Award (2013)
  \item President's Award for Undergraduate Student Excellence (2010), \$1000
  \item State University of New York Chancellor's Award (2010)
  \item John L. Fuller Memorial Award (2010), \$100
  \item Binghamton University Undergraduate Research Award (2008, 2009), \$169 \& \$250
  \item Edward Thorsen Memorial Scholarship (2008), \$400
  \item Freshman Chemistry Award (Fall 2006, Spring 2007), \$50 each
  \item Brain Awareness Week Travel Award (2009), \$750 + conference registration
  \item Sigma Xi Undergraduate Research Poster Award (2008, 2009), \$100 each
  \item U.S. Department of Energy Summer Undergraduate Lab Internship (Summer 2009), \$3200 stipend
  \item Plainview Diner Scholarship (2006), \$500

\end{itemize}

\vspace{0.1in}
{\large \textbf{Professional Mission}}\\
\vspace{0.1in}
My goal is to practice clinical neuropathology and develop an NIH-funded research program to study the molecular mechanisms of neuropsychiatric illness. I am specifically interested in identifying the contribution of discrete neurocircuits to different aspects of behavior, cognition, and disease. Stemming from my doctoral research for instance, I am interested in understanding the distinct roles of two complementary striatal pathways—the striatonigral and striatopallidal pathways—and their contribution to psychosis and affective disorders. Ultimately, I am interested in using single-cell sequencing and neuroanatomical tracing studies for unbiased, high-throughput characterization of neurocircuits. I then want to use post mortem human tissue to investigate the clinical significance of the targets identified from these animal studies. My clinical training in neuropathology will not only maintain my direct involvement in  patient-care, something that I am eager to accomplish, but it will also keep my studies clinically informed and translational.

\end{document}
